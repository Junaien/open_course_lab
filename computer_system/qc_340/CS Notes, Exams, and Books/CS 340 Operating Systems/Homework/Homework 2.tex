\documentclass[12pt]{book}
\usepackage{framed}
\usepackage{lipsum}
\usepackage[dvipsnames]{color}
\usepackage{verbatim}
\usepackage[margin=1.5cm]{geometry}
\usepackage{fancyhdr}
\usepackage{amsfonts}
\usepackage{amssymb}
\usepackage{amsmath}
%\usepackage{graphicx}
\usepackage[all]{xy}
\usepackage{amsthm}
\usepackage{scalefnt}
\usepackage{colonequals}
\usepackage[parfill]{parskip}
\usepackage{moreverb} % for verbatimtab 
\usepackage{hyperref}

\definecolor{shadecolor}{named}{GreenYellow}
\author{Todd Gaugler}

%%I'm using the following font: http://www.tug.dk/FontCatalogue/mathfonts.html
%% comment out unused packages below
%%\usepackage{color}
\usepackage{ifpdf}
\ifpdf %if using pdfLaTeX in PDF mode
  \usepackage[pdftex]{graphicx}
  \DeclareGraphicsExtensions{.pdf,.png,.jpg,.jpeg,.mps}
%  \usepackage{pgf}
  \usepackage{tikz}
\else %if using LaTeX or pdfLaTeX in DVI mode
  \usepackage{graphicx}
  \DeclareGraphicsExtensions{.eps,.bmp}
  \DeclareGraphicsRule{.emf}{bmp}{}{}% declare EMF filename extension
  \DeclareGraphicsRule{.png}{bmp}{}{}% declare PNG filename extension
%  \usepackage{pgf}
  \usepackage{tikz}
  \usepackage{pstricks}
\fi
\usepackage{epic,bez123}
\usepackage{floatflt}% package for floatingfigure environment
\usepackage{wrapfig}% package for wrapfigure environment


\newcommand{\bd}{\textbf}


\newenvironment{code}{ \verbatimtab  }{ \endverbatimtab }
\newenvironment{de}{\noindent \\ \shaded \textbf{Definition} \endshaded  }{\begin{center}
\line(1,0){420}
\end{center} \noindent \\}

\pagestyle{fancy}
\fancyhead{}
\fancyhead[LE]{\slshape \leftmark}
\fancyhead[RO]{\slshape \rightmark}
\fancyfoot[C]{\thepage}

\title{Operating Systems, Homework \#2}
\begin{document}
% Upper part of the page

\maketitle
\section{Go to the root. Get a listing of your root directory.}
\begin{code}
[gato5111@venus ~]$ ls -l
total 236
drwx------  2 gato5111 underg    4096 Jan 30  2012 bin
lrwxrwxrwx  1 gato5111 underg      22 Jan 30  2012 cl -> /home/faculty/ykong/cl
lrwxrwxrwx  1 gato5111 underg      27 Jan 30  2012 clojure -> /home/faculty/ykong/clojure
-rw-------  1 gato5111 underg     663 Nov  2  2007 CS316ex0.java
-rw-------  1 gato5111 underg    1653 Nov  2  2007 CS316ex10.java
-rw-------  1 gato5111 underg     326 Nov  2  2007 CS316ex11.java
\end{code}

\section{List the Contents of the /bin Directory} 
\begin{code} 
[gato5111@venus etc]$ cd /bin
[gato5111@venus bin]$ ls -l
total 8868
-rwxr-xr-x 1 root root    8992 Feb 22  2012 alsacard
-rwxr-xr-x 1 root root   21248 Feb 22  2012 alsaunmute
-rwxr-xr-x 1 root root    7440 Feb 22  2012 arch
lrwxrwxrwx 1 root root       4 Mar 19  2012 awk -> gawk
-rwxr-xr-x 1 root root   20984 Mar 21  2012 basename
-rwxr-xr-x 1 root root  801528 Jul 21  2011 bash
-rwxr-xr-x 1 root root   25248 Mar 21  2012 cat
-rwxr-xr-x 1 root root   42944 Mar 21  2012 chgrp
-rwxr-xr-x 1 root root   40216 Mar 21  2012 chmod
-rwxr-xr-x 1 root root   45584 Mar 21  2012 chown
-rwxr-xr-x 1 root root   71016 Mar 21  2012 cp
-rwxr-xr-x 1 root root  110280 Mar 15  2010 cpio
lrwxrwxrwx 1 root root       4 Nov  4  2010 csh -> tcsh
-rwxr-xr-x 1 root root   36648 Mar 21  2012 cut
\end{code}
I recognize: cut, date, bash, echo, and cat. 

\section{List the device directory}
\begin{code}
[gato5111@venus bin]$ cd /dev
[gato5111@venus dev]$ ls -l
total 0
crw-rw---- 1 root audio  14,   12 Aug  8 19:59 adsp
crw------- 1 root root   10,  175 Aug  8 19:59 agpgart
crw-rw---- 1 root audio  14,    4 Aug  8 19:59 audio
crw------- 1 root root   10,   59 Aug  8 20:00 autofs
drwxr-xr-x 3 root root         60 Aug  8 19:59 bus
crw------- 1 root root    5,    1 Sep  1 04:04 console
lrwxrwxrwx 1 root root         11 Aug  8 19:59 core -> /proc/kcore
drwxr-xr-x 4 root root         80 Aug  8 19:59 cpu
crw------- 1 root root   10,   63 Aug  8 19:59 cpu_dma_latency
drwxr-xr-x 6 root root        120 Aug  8 19:59 disk
drwxr-xr-x 2 root root         60 Aug  8 20:01 dri
crw-rw---- 1 root audio  14,    3 Aug  8 19:59 dsp
lrwxrwxrwx 1 root root         13 Aug  8 19:59 fd -> /proc/self/fd
\end{code}
I recognize: bus, console, cpu, and disk. 

\section{List the Contents of the /etc Directory}

\begin{code}
-rw-r--r--  1 root   root        823 Feb 22  2012 csh.login
drwxr-xr-x  5 root   lp         4096 Mar 19  2012 cups
drwxr-xr-x  4 root   root       4096 Sep 23  2011 dbus-1
drwxr-xr-x  2 root   root       4096 Aug 31 15:09 default
drwxr-xr-x  2 root   root       4096 Oct 15  2010 depmod.d
drwxr-xr-x  2 root   root       4096 Feb 22  2012 desktop-profiles
drwxr-xr-x  3 root   root       4096 Nov 14  2011 dev.d
-rw-r--r--  1 root   root        178 Mar  5  2011 dhcp6c.conf
-rw-r--r--  1 root   root       2518 Mar 21  2012 DIR_COLORS
\end{code}
Files that I have heard of are as follows: profile, protocals, and exports. 
The most used permission is the one indicated by  `-rw-r--r-- '. 
Exploring the passwd file, 
\begin{code}
boda6020:x:4490:800:Daniel B Boadu:/home/fa12/211/boda6020:/bin/bash
doqi5040:x:4491:800:Qiang Dong:/home/fa12/211/doqi5040:/bin/bash
doza1422:x:4492:800:Zachary Donner:/home/fa12/211/doza1422:/bin/bash
bata8837:x:4493:800:Tattiana E Bailey:/home/fa12/211/bata8837:/bin/bash
gojo3689:x:4494:800:Joseph E Golden:/home/fa12/211/gojo3689:/bin/bash
leju3831:x:4495:800:Justin W Leung:/home/fa12/211/leju3831:/bin/bash
\end{code}

\section{Determining the absolute pathname} 
\begin{code}
[gato5111@venus etc]$
[gato5111@venus etc]$ echo $HOME
/home/fa12/340/gato5111
[gato5111@venus etc]$ pwd
/etc
\end{code}
\section{Changing Shells}
\begin{code}
[gato5111@venus etc]$ cat /etc/shells
/bin/sh
/bin/bash
/sbin/nologin
/bin/tcsh
/bin/csh
/bin/ksh
/bin/zsh
/usr/bin/ksh
/bin/pdksh
[gato5111@venus etc]$ chsh
Changing shell for gato5111.
Password:
New shell [/bin/bash]: /bin/tcsh
Shell changed.
[gato5111@venus etc]$ echo $SHELL
/bin/bash
[gato5111@venus etc]$ ps
  PID TTY          TIME CMD
  393 pts/4    00:00:00 bash
  431 pts/4    00:00:00 ps
32516 pts/4    00:00:05 bash
[gato5111@venus etc]$
\end{code}
I observe that there are two instances of `bash' running. 
\begin{code}
[gato5111@venus ~]$ set | more
COLORS /etc/DIR_COLORS.xterm
_ cdpath

addsuffix
argv ()
consoletype pty
csubstnonl
cwd /home/fa12/340/gato5111
dirstack /home/fa12/340/gato5111
echo_style both
edit
gid 800
group underg
histdup erase
history 100
home /home/fa12/340/gato5111
killring 30
mpi_selection
mpi_selector_dir /var/lib/mpi-selector/data
mpi_selector_homefile /home/fa12/340/gato5111/.mpi-selector
mpi_selector_sysfile /etc/sysconfig/mpi-selector
owd /etc
path (/usr/kerberos/bin /usr/local/bin /bin /usr/bin /home/fa12/340/gato
5111/bin)
prompt [gato5111@%m %c]%#
prompt2 %R?
prompt3 CORRECT>%R (y|n|e|a)?
promptchars $#
savehist (1024 merge)
shell /bin/tcsh
shlvl 3
sourced 1
status 0
tcsh 6.14.00
term xterm
tty pts/4
uid 3191
user gato5111
version tcsh 6.14.00 (Astron) 2005-03-25 (x86_64-unknown-linux) options wid
e,nls,dl,al,kan,sm,rh,color,filec
\end{code}
in viewing the environment variables for the shell, we have
 \begin{code}

[gato5111@venus ~]$ env
HOSTNAME=venus
SHELL=/bin/bash
TERM=xterm
HISTSIZE=1000
SSH_CLIENT=149.4.115.3 50101 22
SSH_TTY=/dev/pts/4
USER=gato5111
LS_COLORS=no=00:fi=00:di=00;34:ln=00;36:pi=40;33:so=00;35:bd=40;33;01:cd=40;33;01:or=01;05;37;41:mi=01;05;37;41:ex=00;32:*.cmd=00;32:*.exe=00;32:*.com=00;32:*.btm=00;32:*.bat=00;32:*.sh=00;32:*.csh=00;32:*.tar=00;31:*.tgz=00;31:*.arj=00;31:*.taz=00;31:*.lzh=00;31:*.zip=00;31:*.z=00;31:*.Z=00;31:*.gz=00;31:*.bz2=00;31:*.bz=00;31:*.tz=00;31:*.rpm=00;31:*.cpio=00;31:*.jpg=00;35:*.gif=00;35:*.bmp=00;35:*.xbm=00;35:*.xpm=00;35:*.png=00;35:*.tif=00;35:
PATH=/usr/kerberos/bin:/usr/local/bin:/bin:/usr/bin:/home/fa12/340/gato5111/bin
MAIL=/var/spool/mail/gato5111
PWD=/home/fa12/340/gato5111
INPUTRC=/etc/inputrc
LANG=en_US.UTF-8
SSH_ASKPASS=/usr/libexec/openssh/gnome-ssh-askpass
HOME=/home/fa12/340/gato5111
SHLVL=3
LOGNAME=gato5111
CVS_RSH=ssh
SSH_CONNECTION=149.4.115.3 50101 149.4.211.180 22
LESSOPEN=|/usr/bin/lesspipe.sh %s
G_BROKEN_FILENAMES=1
_=/bin/csh
HOSTTYPE=x86_64-linux
VENDOR=unknown
OSTYPE=linux
MACHTYPE=x86_64
GROUP=underg
HOST=venus
REMOTEHOST=bsc.qc.cuny.edu
\end{code}

\begin{code}
[gato5111@venus ~]$ man setenv
SETENV(3)                  Linux Programmerâs Manual                 SETENV(3)

NAME
       setenv - change or add an environment variable

SYNOPSIS
       #include <stdlib.h>

       int setenv(const char *name, const char *value, int overwrite);

       int unsetenv(const char *name);

DESCRIPTION
       The setenv() function adds the variable name to the environ-
       ment with the value value, if name does not  already  exist.
       If  name  does  exist  in the environment, then its value is
       changed to value if overwrite is non-zero; if  overwrite  is
       zero, then the value of name is not changed.

       The  unsetenv()  function deletes the variable name from the
       environment.

\end{code}

\begin{code}
[gato5111@venus ~]$ setenv | more
HOSTNAME=venus
SHELL=/bin/bash
TERM=xterm
HISTSIZE=1000
SSH_CLIENT=149.4.115.3 50101 22
SSH_TTY=/dev/pts/4
USER=gato5111
LS_COLORS=no=00:fi=00:di=00;34:ln=00;36:pi=40;33:so=00;35:bd=40;33;01:cd=40
;33;01:or=01;05;37;41:mi=01;05;37;41:ex=00;32:*.cmd=00;32:*.exe=00;32:*.com
=00;32:*.btm=00;32:*.bat=00;32:*.sh=00;32:*.csh=00;32:*.tar=00;31:*.tgz=00;
31:*.arj=00;31:*.taz=00;31:*.lzh=00;31:*.zip=00;31:*.z=00;31:*.Z=00;31:*.gz
=00;31:*.bz2=00;31:*.bz=00;31:*.tz=00;31:*.rpm=00;31:*.cpio=00;31:*.jpg=00;
35:*.gif=00;35:*.bmp=00;35:*.xbm=00;35:*.xpm=00;35:*.png=00;35:*.tif=00;35:
PATH=/usr/kerberos/bin:/usr/local/bin:/bin:/usr/bin:/home/fa12/340/gato5111
\end{code}

\section{Processes} 
\begin{code}
[gato5111@venus ~]$ man ps
PS(1)                         Linux Userâs Manual                        PS(1)

NAME
       ps - report a snapshot of the current processes.

SYNOPSIS
       ps [options]

DESCRIPTION
       ps displays information about a selection of the active
       processes. If you want a repetitive update of the selection
       and the displayed information, use top(1) instead.

       This version of ps accepts several kinds of options:
       1   UNIX options, which may be grouped and must be preceded
           by a dash.
       2   BSD options, which may be grouped and must not be used
           with a dash.
       3   GNU long options, which are preceded by two dashes.

       Options of different types may be freely mixed, but
       conflicts can appear. There are some synonymous options,
       which are functionally identical, due to the many standards
       and ps implementations that this ps is compatible with.

\end{code}
\begin{code}
Process states in Unix are: 
R - runnable which means the process has done a context switch and has the kernel.
S - sleeping which means the process is waiting on I/O completion (blocked), a pipe, memory, etc.
T - process has been stopped - sent a SIGSTOP usually with ctrl/z
Z - zombie - a process that has a process image in memory but no context, ie., not swappable.
(-google)

D – 3 – 
F – is the flags 
S – Process of the status code
UID – Username of the process’s owner
PID – Process ID number
PPID – ID number of process’s parents process
C – SPU usage and scheduling information
PRI – Priority of the process
NI – Nice value 
ADDR – Memory address of the process
SZ – Virutal memory usage
WCHAN – Memory address of the event the process is waiting for
TTY – Terminal associated with the process
TIME – Total CPU usage
CMD – name of the process

\end{code}

\begin{code}
[gato5111@venus ~]$ ps -l
F S   UID   PID  PPID  C PRI  NI ADDR SZ WCHAN  TTY          TIME CMD
0 S  3191   393 32516  0  75   0 - 16524 wait   pts/4    00:00:00 bash
0 S  3191   525   393  0  75   0 - 17733 rt_sig pts/4    00:00:00 csh
0 R  3191  2548   525  0  76   0 - 15883 -      pts/4    00:00:00 ps
0 S  3191 32516 32515  0  75   0 - 16550 wait   pts/4    00:00:05 bash
[gato5111@venus ~]$
\end{code}
\begin{code}top - 18:16:50 up 46 days, 22:17, 13 users,  load average: 0.00, 0.00, 0.0
Tasks: 188 total,   1 running, 187 sleeping,   0 stopped,   0 zombie
Cpu(s):  0.2%us,  0.0%sy,  0.0%ni, 99.8%id,  0.0%wa,  0.0%hi,  0.0%si,  0.
Mem:   3967152k total,  3094912k used,   872240k free,   212428k buffers
Swap:  4104596k total,   614028k used,  3490568k free,  2122540k cached

  PID USER      PR  NI  VIRT  RES  SHR S %CPU %MEM    TIME+  COMMAND
    1 root      15   0 10364   88   56 S  0.0  0.0   0:13.22 init
    2 root      RT  -5     0    0    0 S  0.0  0.0   0:00.00 migration/0
    3 root      34  19     0    0    0 S  0.0  0.0   0:00.14 ksoftirqd/0
    4 root      RT  -5     0    0    0 S  0.0  0.0   0:00.00 watchdog/0
    5 root      RT  -5     0    0    0 S  0.0  0.0   0:00.37 migration/1
    6 root      34  19     0    0    0 S  0.0  0.0   0:00.20 ksoftirqd/1
    7 root      RT  -5     0    0    0 S  0.0  0.0   0:00.00 watchdog/1
    8 root      10  -5     0    0    0 S  0.0  0.0   0:00.10 events/0
    9 root      10  -5     0    0    0 S  0.0  0.0   0:00.11 events/1
   10 root      10  -5     0    0    0 S  0.0  0.0   0:00.03 khelper
   51 root      11  -5     0    0    0 S  0.0  0.0   0:00.00 kthread
   56 root      10  -5     0    0    0 S  0.0  0.0   0:00.24 kblockd/0
   57 root      10  -5     0    0    0 S  0.0  0.0   0:01.38 kblockd/1
   58 root      14  -5     0    0    0 S  0.0  0.0   0:00.00 kacpid
  130 root      11  -5     0    0    0 S  0.0  0.0   0:00.00 cqueue/0
  131 root      11  -5     0    0    0 S  0.0  0.0   0:00.00 cqueue/1
  134 root      10  -5     0    0    0 S  0.0  0.0   0:00.00 khubd
  136 root      10  -5     0    0    0 S  0.0  0.0   0:00.14 kseriod
  210 root      15   0     0    0    0 S  0.0  0.0   0:00.00 khungtaskd
  213 root      10  -5     0    0    0 S  0.0  0.0   1:44.09 kswapd0
  214 root      12  -5     0    0    0 S  0.0  0.0   0:00.00 aio/0
  215 root      12  -5     0    0    0 S  0.0  0.0   0:00.00 aio/1
  361 root      11  -5     0    0    0 S  0.0  0.0   0:00.00 kpsmoused
  393 gato5111  15   0 66096 1536 1204 S  0.0  0.0   0:00.03 bash
  395 root      15  -5     0    0    0 S  0.0  0.0   0:00.00 ata/0
  396 root      16  -5     0    0    0 S  0.0  0.0   0:00.00 ata/1
  397 root      15  -5     0    0    0 S  0.0  0.0   0:00.00 ata_aux
  401 root      10  -5     0    0    0 S  0.0  0.0   0:00.00 scsi_eh_0
  402 root      10  -5     0    0    0 S  0.0  0.0   0:00.00 scsi_eh_1
  403 root      10  -5     0    0    0 S  0.0  0.0   0:00.00 scsi_eh_2
  404 root      10  -5     0    0    0 S  0.0  0.0   0:00.00 scsi_eh_3
  420 root      11  -5     0    0    0 S  0.0  0.0   0:00.00 kstriped
  432 nobody    15   0  184m 4576 2760 S  0.0  0.1   0:00.02 httpd
  433 root      10  -5     0    0    0 S  0.0  0.0   0:41.40 kjournald
  451 root      16   0 90304 3476 2580 S  0.0  0.1   0:00.06 sshd
  458 bere9237  15   0 90304 2000 1100 S  0.0  0.1   0:00.13 sshd
  459 root      10  -5     0    0    0 S  0.0  0.0   0:00.93 kauditd
  460 bere9237  15   0 66096 1600 1200 S  0.0  0.0   0:00.03 bash
\end{code}
\begin{code}
[gato5111@venus ~]$ man fork
FORK(2)                    Linux Programmerâs Manual                   FORK(2)

NAME
       fork - create a child process

SYNOPSIS
       #include <sys/types.h>
       #include <unistd.h>

       pid_t fork(void);

DESCRIPTION
       fork()  creates a child process that differs from the parent
       process only in its PID and  PPID,  and  in  the  fact  that
       resource  utilizations are set to 0.  File locks and pending
       signals are not inherited.

       Under  Linux,  fork()  is  implemented  using  copy-on-write
       pages,  so  the  only penalty that it incurs is the time and
       memory required to duplicate the parentâs page  tables,  and
       to create a unique task structure for the child.



[gato5111@venus ~]$ man execve
EXECVE(2)                  Linux Programmerâs Manual                 EXECVE(2)

NAME
       execve - execute program

SYNOPSIS
       #include <unistd.h>

       int execve(const char *filename, char *const argv[],
                  char *const envp[]);

DESCRIPTION
       execve() executes the program pointed to by filename.  file-
       name must be either a binary executable, or a script  start-
       ing  with a line of the form "#! interpreter [arg]".  In the
       latter case, the interpreter must be a valid pathname for an
       executable  which  is  not  itself  a  script, which will be
       invoked as interpreter [arg] filename.

       argv is an array of argument strings passed to the new  pro-
       gram.   envp  is  an array of strings, conventionally of the
       form key=value, which are passed as environment to  the  new
       program.   Both  argv  and envp must be terminated by a null
       pointer.   The  argument  vector  and  environment  can   be
       accessed  by  the called programâs main function, when it is
       defined as int main(int argc, char *argv[], char *envp[]).

   
[gato5111@venus ~]$ man wait
BASH_BUILTINS(1)                                              BASH_BUILTINS(1)

NAME
       bash, :, ., [, alias, bg, bind, break, builtin, cd, command,
       compgen, complete, continue, declare,  dirs,  disown,  echo,
       enable,  eval,  exec,  exit,  export, fc, fg, getopts, hash,
       help, history, jobs, kill, let, local, logout, popd, printf,
       pushd,  pwd,  read,  readonly,  return,  set,  shift, shopt,
       source, suspend, test, times, trap, type,  typeset,  ulimit,
       umask,  unalias,  unset,  wait - bash built-in commands, see
       bash(1)

BASH BUILTIN COMMANDS
       Unless otherwise noted, each builtin command  documented  in
       this  section  as accepting options preceded by - accepts --
       to signify the end of the  options.   For  example,  the  :,
       true, false, and test builtins do not accept options.  Also,
       please note that while executing in non-interactive mode and
       while  in posix mode, any special builtin (like ., :, break,


[gato5111@venus ~]$ man kill
KILL(1)                    Linux Programmerâs Manual                   KILL(1)

NAME
       kill - terminate a process

SYNOPSIS
       kill [ -s signal | -p ] [ -a ] [ -- ] pid ...
       kill -l [ signal ]

DESCRIPTION
       The command kill sends the specified signal to the specified
       process or process group.  If no signal  is  specified,  the
       TERM  signal  is  sent.  The TERM signal will kill processes
       which do not catch this signal.  For other processes, it may
       be  necessary  to use the KILL (9) signal, since this signal

\end{code}


\section{Windows CreateProcess()}
CreateProcess()
Applies to: desktop apps only
Creates a new process and its primary thread. The new process runs in the security context of the calling process.
If the calling process is impersonating another user, the new process uses the token for the calling process, not the impersonation token. To run the new process in the security context of the user represented by the impersonation token, use the CreateProcessAsUser or CreateProcessWithLogonW function.


\section{Running Processes}
\begin{code}
[gato5111@venus ~]$ gcc parent.c -o parent
parent.c: In function âmainâ:
parent.c:8: warning: incompatible implicit declaration of built-in function âexitâ
parent.c:11: warning: incompatible implicit declaration of built-in function âprintfâ
[gato5111@venus ~]$ gcc child.c -o child
child.c:2: error: expected declaration specifiers or â...â before string constant
child.c:2: error: expected declaration specifiers or â...â before âgetpidâ
child.c:2: warning: data definition has no type or storage class
child.c:2: warning: conflicting types for built-in function âprintfâ
child.c:3: error: expected identifier or â(â before â}â token
child.c:3: error: expected â=â, â,â, â;â, âasmâ or â__attribute__â before âPICOâ
[gato5111@venus ~]$ ./parent
Process[3587]: Parent in execution ...
Process[3587]: Parent detects terminating child
Process[3587]: Parent terminating ...
[gato5111@venus ~]$ gcc orphan.c -o orphan
[gato5111@venus ~]$ ./orphan
I'm the original process with PID 3888 and PPID 3543.
I'm the parent process with PID 3888 and PPID 3543.
my child's PID 3889
PID 3888 terminates.
[gato5111@venus ~]$
\end{code}

Although there isn't much output to comment on, I'll comment on what I have: based on the definitions of parent.c, orphan.c, and child.c it makes sense that after running and compiling parent, then child, that running orphan would have the PID as the parent process, and that it terminates after correctly identifying the child processes run before it. 



\end{document}